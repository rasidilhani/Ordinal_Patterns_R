\chapter{Introduction}\label{C:intro}

Time series contain valuable insights about the underlying system that generates the data. 
Their analysis is typically conducted using two primary approaches: time-domain and transformed-domain methods. 
In the context of time-domain analysis, Bandt and Pompe \cite{PhysRevLett.88.174102} introduced a novel methodology that is non-parametric and rooted in information theory descriptors: Ordinal Patterns symbolisation. 

Bandt and Pompe \cite{PhysRevLett.88.174102} proposed transforming small subsets of the time series observations into symbols that encode the sorting properties of the values in these subsets.
Then, they computed a histogram of those symbols.
The resulting distribution is less sensitive to outliers compared to the original data, and the histogram is independent of any specific model.
The proposal proceeds by computing descriptors from this histogram, and extracting information about the system from these descriptors.
As a result, this approach is versatile and applicable to a wide range of scenarios.  

This study explores features derived from Bandt and Pompe symbolization, specifically Shannon entropy, and represents them graphically as a point in the entropy-complexity plane under the Multinomial distribution, which is represented as a point in the $\mathbb{R}^2$ manifold. 
%%% ACF Is the Shannon Entropy a point in the HxC plane?
%%% ACF Is that true under the Multinomial distribution?
Furthermore, this proposal separately examines the permutation entropy and ordinal patterns in time series analysis, including the calculation of pattern histograms, entropy, complexity, and confidence intervals for entropy and complexity to enhance the understanding of their statistical properties. 
The confidence intervals for entropy and complexity under the Multinomal distribution and features for time series clustering will be discussed further. 
Future work will extend to other measures, such as Rényi entropy, Fisher information, and the confidence intervals for their entropy and complexity under the Multinomal distribution. Features for time series clustering with ordinal patterns for time series analysis will be further studied.

Time series analysis is widely applied across various fields, including engineering, economics, physical sciences, and more. A time series is defined as a collection of observations ${x_t}$, each representing a realized value of a particular random variable $X_t$, where time can be either discrete or continuous.

Examples of time series applications include finance (e.g., analyzing exchange rate movements or commodity prices), biology (e.g., modeling the growth and decline of bacterial populations), medicine (e.g., tracking the spread of diseases like COVID-19 or influenza), and geoscience (e.g., predicting wet or dry days based on past weather conditions).

The primary goal of time series analysis is to understand the nature of the phenomenon represented by the observed sequence. Time domain and frequency domain methods are the two primary approaches used in time series analysis. The temporal approach relies on concepts such as auto-correlation and regressions, where a time series' present value is analyzed in relation to its own past values or the past values of other series. This method represents time series directly as a function of time. On the other hand, the spectral approach represents time series through spectral expansions, such as wavelets or Fourier modes~\cite{treitel1995spectral}.

 
However, these methods often require assumptions such as large sample sizes or normally distributed observations that are rarely met in real-world empirical data. For many statistical techniques to be valid, these assumptions must hold, but in practice, they are frequently violated.

For example, traditional approaches to time series analysis, such as time domain and frequency domain methods, rely on assumptions that are not always valid in real-world data. The time domain approach, which uses techniques like auto-correlation and regression, assumes stationarity and often struggles with nonlinear or nonstationary data. Similarly, the frequency domain approach, which represents time series through spectral expansions such as wavelets or Fourier modes, may require assumptions about periodicity and may not effectively capture short-term fluctuations.

Many statistical methods in these approaches depend on specific conditions, such as large sample sizes or normally distributed observations. However, these assumptions are often unrealistic, leading to inaccurate or biased results. When such conditions are not met, alternative methods must be considered.

As a result, alternative methods, often referred to as non-parametric techniques, must be considered. 
These methods rely on the rank $R_t$ of the observations $x_t$ rather than their actual values, making them robust and applicable to a wide range of data sets. Since non-parametric tests do not assume a normal distribution, they are highly reliable. For example, the Kruskal-Wallis $H$ test and the Wilcoxon test are effective tools for comparing two or more population probability distributions from independent random samples. However, these techniques are not always suitable for time series data, which often require specialized methods tailored to their unique characteristics.

To address these challenges, ordinal pattern methods provide a robust alternative. Instead of analyzing the absolute values of a time series, these methods focus on the order relationships among consecutive data points.

This approach effectively captures the underlying dynamics of complex systems and offers several advantages.

The ordinal pattern-based method has become a widely used tool for characterizing complex time series. Since its introduction nearly twenty-three years ago by Bandt and Pompe in their foundational paper \cite{PhysRevLett.88.174102}, it has been successfully applied across various scientific fields, including biomedical signal processing, optical chaos, hydrology, geophysics, econophysics, engineering, and biometrics. It has also been used in the characterization of pseudo-random number generators.
   
The Bandt and Pompe method successfully analyzes time series by transforming them into ordinal patterns, constructing a histogram, and computing Shannon entropy, making it robust against outliers and independent of predefined models.

Later, Rosso~\cite{Rosso2007} introduced an additional dimension to this analysis Statistical Complexity derived from the same histogram of causal patterns.

\section*{Introduction to Ordinal Pattern Analysis}

Ordinal patterns are a non-parametric representation of real-valued time series and ordinal patterns are transformations that encode the sorting characteristics of values in $\mathbb{R}^D$ into $D!$ symbols, where $D$ represents for the ''Embedding Dimension'' and usually ranges between three to six. 
One of the possible encoding is the set of indexes that sort the $D$ values in non-decreasing order.
 

To illustrate this idea, let $\bm{x}=\{x_1,x_2, \dots, x_{(n+D-1)}\}$ 
be a real valued time series of length $n+D-1$ without ties. 
As stated by Bandt \& Pompe, if the $\bm{x}$ takes infinitely many values, it is common to replace them with a symbol sequence $\bm{{\pi}}=({\pi}_1, {\pi}_2,\dots, {\pi}_n)$.
consisting of finitely many symbols and then compute the entropy from this sequence. 
The corresponding symbol sequence naturally emerges from the time series without requiring any model assumptions. We compute
$\bm{{\pi}}=({\pi}_1, {\pi}_2,\dots, {\pi}_n)$ symbols from sub-sequences of embedding dimension $D$. 
There are $D!$ possible symbols: $\pi_j \in \bm{{\pi}}=({\pi}^1, {\pi}^2,\dots, {\pi}^{D!})$. 
The histogram of proportions $h=(h_1,h_2,\dots, h_{D!})$ in which the bin $h_\ell$ 
%%% ACF Whenever possible, use \ell instead of the letter ell in mathematical mode
is the proportion of symbols of type $\pi^\ell$ of the total number of symbols. 
For convenience, we will model those symbols as a $k$ dimensional random vector where $k=D!$.

Consider a series of $n$ independent trials in which only one of $k$ mutually exclusive events ${\pi}^1, {\pi}^2,\dots, {\pi}^k$ is observed with probability $p_1, p_2, \dots, p_k,$ respectively, where $p_\ell \geq 0$ and $\sum_{\ell=1}^{k} p_\ell=1$. 
Suppose that $N=(N_1, N_2, \dots, N_k)$ is the vector of random variables that, with $\sum_{\ell=1}^{k} N_\ell=n$, counts how many times the events ${\pi}^1, {\pi}^2,\dots, {\pi}^k$ occur in the $n$ trials. Then, the joint probability distribution of $N$ is
%%% ACF NEVER leave a blank line between the paragraph and the equation that follows
\begin{equation}
	\Pr\big(N=(n_1,n_2,\dots, n_k)\big) = n! \prod_{\ell=1}^{k} \frac{p_\ell^{n_\ell}}{n_\ell !}, %%% ACF Propagate the changes
\end{equation}    
where $n_\ell \geq 0$ and $\sum_{\ell=1}^{k} n_\ell=n$ .


\section*{Problem Statement}
%To illustrate this concept, we consider food preferences among individuals as an example of time series data. Meal choices vary widely among people, reflecting their unique tastes and priorities. For instance, when selecting main meals such as beef, pork, chicken, fish, and vegetables, each person has distinct preferences. By analyzing these choices, we can uncover intriguing patterns that highlight the diversity in individual selections.


To illustrate this concept, imagine tracking the mean monthly humidity in Wellington. You want to analyze how humidity changes throughout the year. By examining this data, you can uncover interesting patterns that highlight the variations in humidity across different months. 

%Imagine two individuals with different meal preferences. Person 1 enjoys pork the most, followed by beef, fish, chicken, and vegetables. In contrast, Person 2 prefers chicken first, followed by fish, vegetables, beef, and pork. A simple graph illustrating these preferences would show that their choices do not overlap significantly, emphasizing their unique tastes.

%Expanding this analysis to a larger group allows us to observe even more diverse and complex patterns. Each individual's preference forms a unique data point, and collectively, they create a rich tapestry of variation. This exploration provides valuable insights into how food choices differ across individuals and groups, making the study both meaningful and captivating.

\begin{table}[hbt]
	\centering
	\begin{tabular}{lc}
		\toprule
		 Month & Mean of relative humidity \\
		\midrule
		January & 77.3 \\ 
		February & 81 \\
		March & 82.4 \\
		April & 81.7 \\
		May & 83.6 \\ 
		June & 85.6 \\
		July & 84.4 \\
		August & 83.1 \\ 
		September & 78.8 \\
		October & 79.6 \\
		November & 78.2 \\
		December & 78.8 \\
		\bottomrule
	\end{tabular}
	\caption{Mean monthly humidity variations in Wellington throughout the year}
	\label{tab:humidity}
\end{table}

Mean monthly humidity in Wellington is shown in Figure~\ref{fig:humidity}  
\begin{figure}[hbt]
	\centering
	\includegraphics[width=0.6\textwidth]{humidity graph}
	\caption{Mean Monthly humidity in Wellington}
	\label{fig:humidity}
\end{figure}

We can convert this actual data into ordinal patterns. 
To do this, for each month, we determine the order of the humidity values rather than their actual magnitudes. 
Each three-time-point sequence (which can be adjusted based on preference) is converted into an ordinal pattern.
This ``embedding dimension'' usually varies between $3$ and $6$, but any dimension is possible.
The conversion can be made in any way that uniquely maps the sorting properties of the subsequence into a symbol.
%%% ACF Summarise and present in your own words from Section 3
%@Article\{AsymptoticDistributionofEntropiesandFisherInformationMeasureofOrdinalPatternswithApplicationsa,
%author    = \{Rey, A. A. and Frery, A. C. and Gambini, J. and Lucini, M.\},
%journal   = \{Chaos, Solitons \textbackslash{}\& Fractals\},
%title     = \{Asymptotic distribution of entropies and \{F\}isher information measure of ordinal patterns with applications\},
%year      = \{2024\},
%issn      = \{0960-0779\},
%month     = nov,
%pages     = \{115481\},
%volume    = \{188\},
%doi       = \{10.1016/j.chaos.2024.115481\},
%groups    = \{Journal Article\},
%publisher = \{Elsevier BV\},
%\}


%%% ACF Use only the lines provided by the booktabs package
%%% ACF The table is confusing: "# of patterns" should be just $t$, Ordinal Patters should be Ordinal pattern, and you should assign a type to each, i.e., \pi^1, \pi^2 etc

\begin{table}[H]
	\centering
	\begin{tabular}{lcr}
		\toprule
		 $t$ & Mean Humidity sequence & Ordinal Pattern \\
		\midrule
	1 & (77.3,81,82.4) & (123) $=\pi^1$\\ 
	2 & (81,82.4,81.7) & (132) $=\pi^2$\\
	3 & (82.4,81.7,83.6) & (213) $=\pi^3$\\
	4 & (81.7,83.6,85.6) & (123) $=\pi^1$ \\
	5 & (83.6,85.6,84.4) & (132) $=\pi^2$\\ 
	6 & (85.6,84.4,83.1) & (321) $=\pi^6$\\
	7 & (84.4,83.1,78.8) & (321) $=\pi^6$\\
	8 & (83.1,78.8,79.6) & (312) $=\pi^5$\\ 
	9 & (78.8,79.6,78.2) & (231) $=\pi^4$\\
	10 & (79.6,78.2,78.8) & (312) $=\pi^5$\\
		\bottomrule
	\end{tabular}
	\caption{Ordinal Patterns}
	\label{tab:op}
\end{table}

As shown in Table~\ref{tab:op}, we have six mutually exclusive events,   
$({\pi}^1, {\pi}^2,\dots, {\pi}^{6})= {(123),(132),(213),(231),(312),(321)}$ respectively. 
The probability distribution of the mean humidity is calculated based on ordinal patterns as given below.
\begin{equation}
	\hat{p_i}=\frac{\#\{\pi_j \in \bm{\pi}: \pi_j=\pi^{i}\}}{n}; 1\le i \le 6
\end{equation}
where $\hat{\bm{p}}=(\hat{p_1}, \dots ,\hat{p_6})$

\begin{table}[hbt]
	\centering
	\begin{tabular}{lc}
		\toprule
		Notation & Probability \\ 
		\midrule
		$p(\pi^1)$ & $\frac{2}{10}$ \\ 
		$p(\pi^2)$ & $\frac{2}{10}$ \\
		$p(\pi^3)$ & $\frac{1}{10}$ \\
		$p(\pi^4)$ & $\frac{1}{10}$ \\
		$p(\pi^5)$ & $\frac{2}{10}$ \\ 
		$p(\pi^6)$ & $\frac{2}{10}$ \\
		\bottomrule
	\end{tabular}
	\caption{Probability function}
	\label{tab:probability_function}
\end{table}

We construct the histogram of proportions $h=(h_1,h_2,h_3,h_4,h_5,h_6)$, where each bin $h_\ell$ represents the proportion of symbols of type $\pi^\ell$ out of the total six symbols. The histogram graph is shown Figure~\ref{fig:histogram}.

\begin{figure}[hbt]
	\centering
	\includegraphics[width=0.6\textwidth]{frequency histogram}
	\caption{Histogram of proportions of the observed patterns according to Table~\ref{tab:probability_function}.}
	\label{fig:histogram}
\end{figure}
%%% ACF Always add information as in the following line
%%% The above plot was produced by lines 41-57 of file 'Code/R/humidity_example.R'

This example explains how time series data can be converted into ordinal patterns and how the probability distribution function can be calculated from these patterns. Chapter Two will review the literature on ordinal pattern analysis, while Chapter Three will expand on this concept by exploring the characterization of time series. It will also cover the computation of two key descriptors entropy and complexity from the resulting histograms. Additionally, Chapter Three will outline the main ideas and objectives of this research project.      






