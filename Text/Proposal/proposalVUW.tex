\documentclass[12pt, oneside, a4paper]{book}

\usepackage[pdftex]{graphicx}
\usepackage{tikz-cd}
\usepackage{bm}
\usepackage{amsmath}
\usepackage{amssymb}
\usepackage{amsthm}
\usepackage{pgfplots}
\usepackage{xcolor}
\usepackage[shortlabels]{enumitem}
\usepackage[mathscr]{euscript}
\pgfplotsset{compat = newest}
\usepackage{float}
\usepackage{microtype} 

\usepackage{vuwproposal} % sets up some local things, mostly the front page

\usepackage{palatino} % sets palatino as the default font

\usepackage{url} % for typesetting urls

\usepackage{booktabs} % for table creation



%\renewcommand{\baselinestretch}{1.00}


\begin{document} 
	
	
\frontmatter
% Book style knows about front matter
% Report style doesn't so you need to set roman numbering etc yourself :-(

%%%%%%%%%%%%%%%%%%%%%%%%%%%%%%%%%%%%%%%%%%%%%%%%%%%%%%%
	
\title{New Features of Permutation Entropy in Ordinal Patterns Complexity Plane}
%%% ACF Correct the title

\author{Rasika Dilhani}
\supervisor{Alejandro C.\ Frery}
	
\subject{Data Science}
\abstract{Time series analysis plays a vital role in understanding the underlying dynamics of complex systems across various domains such as engineering, economics, and the physical sciences. Traditional approaches, namely time-domain and frequency-domain methods often rely on strong assumptions such as stationarity, large sample sizes, or normality, which are frequently violated in real-world data. As a robust alternative, Bandt and Pompe~\cite{PhysRevLett.88.174102} introduced a non-parametric method based on ordinal pattern symbolization and information theoretic descriptors. This approach transforms local segments of the time series into rank-based symbols, constructs a histogram of ordinal patterns, and computes permutation entropy, offering resistance to noise and model independence. 
%%% ACF 1- symbolisation, 2- histogram, 3- descriptors (H and C)	
Building on this, Lamberti et al~\cite{lamberti2004intensive} introduced statistical complexity, allowing the joint representation of entropy and complexity in the entropy-complexity plane. 
This proposal investigates the use of permutation entropy and statistical complexity for time series analysis, focusing on pattern histogram 
%%% ACF What do we do with respect to histogram construction?
construction, descriptor estimation, and confidence interval evaluation. The goal is to assess the statistical properties and practical utility of these measures for analyzing nonlinear, noisy, and nonstationary time series data.} 


% Books don't normally have abstracts, and this is a bit of a hack
	
% Uncomment the appropriate degree
\phd
%\mscthesisonly
%\mscwithhonours
%\mscbothparts
% \otherdegree{DEGREE OR DIPLOMA NAME}
	
	
	
%%%%%%%%%%%%%%%%%%%%%%%%%%%%%%%%%%%%%%%%%%%%%%%%%%%%%%%
%New commands:
	
\newcommand{\Sha}{\rotatebox[origin=c]{180}{$\Pi\kern-0.361em\Pi$}}
\newcommand{\ot}{\otimes}
\newtheorem{definition}{Definition}
\newtheorem{theo}{Theorem}
\newtheorem{contoh}{Example}
\newtheorem{cor}{Corollary}
\newtheorem{prop}{Proposition}
\newtheorem{lemma}{Lemma}
\newtheorem{remark}{Remark}
\newtheorem{conjecture}{Conjecture}
%%%%%%%%%%%%%%%%%%%%%%%%%%%%%%%%%%%%%%%%%%%%%%%%%%%%%%%
	
\maketitle
	
%\include{thesis-acknowledge}
	
\tableofcontents
	
	
%%%%%%%%%%%%%%%%%%%%%%%%%%%%%%%%%%%%%%%%%%%%%%%%%%%%%%%
	
% book style knows about mainmatter
% if you are using report style you will have to rest page numbering etc.
\mainmatter
	
%%%%%%%%%%%%%%%%%%%%%%%%%%%%%%%%%%%%%%%%%%%%%%%%%%%%%%%
	
% individual chapters included here
	
\chapter{Introduction}\label{C:intro}

Time series contain valuable insights about the underlying system that generates the data. 
Their analysis is typically conducted using two primary approaches: time-domain and transformed-domain methods. 
In the context of time-domain analysis, Bandt and Pompe \cite{PhysRevLett.88.174102} introduced a novel methodology that is non-parametric and rooted in information theory descriptors: Ordinal Patterns symbolisation. 

Bandt and Pompe \cite{PhysRevLett.88.174102} proposed transforming small subsets of the time series observations into symbols that encode the sorting properties of the values in these subsets.
Then, they computed a histogram of those symbols.
The resulting distribution is less sensitive to outliers compared to the original data, and the histogram is independent of any specific model.
The proposal proceeds by computing descriptors from this histogram, and extracting information about the system from these descriptors.
As a result, this approach is versatile and applicable to a wide range of scenarios.  

This study explores features derived from Bandt and Pompe symbolization, specifically Shannon entropy, and represents them graphically as a point in the entropy-complexity plane under the Multinomial distribution, which is represented as a point in the $\mathbb{R}^2$ manifold. 
%%% ACF Is the Shannon Entropy a point in the HxC plane?
%%% ACF Is that true under the Multinomial distribution?
Furthermore, this proposal separately examines the permutation entropy and ordinal patterns in time series analysis, including the calculation of pattern histograms, entropy, complexity, and confidence intervals for entropy and complexity to enhance the understanding of their statistical properties. 
The confidence intervals for entropy and complexity under the Multinomal distribution and features for time series clustering will be discussed further. 
Future work will extend to other measures, such as Rényi entropy, Fisher information, and the confidence intervals for their entropy and complexity under the Multinomal distribution. Features for time series clustering with ordinal patterns for time series analysis will be further studied.

Time series analysis is widely applied across various fields, including engineering, economics, physical sciences, and more. A time series is defined as a collection of observations ${x_t}$, each representing a realized value of a particular random variable $X_t$, where time can be either discrete or continuous.

Examples of time series applications include finance (e.g., analyzing exchange rate movements or commodity prices), biology (e.g., modeling the growth and decline of bacterial populations), medicine (e.g., tracking the spread of diseases like COVID-19 or influenza), and geoscience (e.g., predicting wet or dry days based on past weather conditions).

The primary goal of time series analysis is to understand the nature of the phenomenon represented by the observed sequence. Time domain and frequency domain methods are the two primary approaches used in time series analysis. The temporal approach relies on concepts such as auto-correlation and regressions, where a time series' present value is analyzed in relation to its own past values or the past values of other series. This method represents time series directly as a function of time. On the other hand, the spectral approach represents time series through spectral expansions, such as wavelets or Fourier modes~\cite{treitel1995spectral}.

 
However, these methods often require assumptions such as large sample sizes or normally distributed observations that are rarely met in real-world empirical data. For many statistical techniques to be valid, these assumptions must hold, but in practice, they are frequently violated.

For example, traditional approaches to time series analysis, such as time domain and frequency domain methods, rely on assumptions that are not always valid in real-world data. The time domain approach, which uses techniques like auto-correlation and regression, assumes stationarity and often struggles with nonlinear or nonstationary data. Similarly, the frequency domain approach, which represents time series through spectral expansions such as wavelets or Fourier modes, may require assumptions about periodicity and may not effectively capture short-term fluctuations.

Many statistical methods in these approaches depend on specific conditions, such as large sample sizes or normally distributed observations. However, these assumptions are often unrealistic, leading to inaccurate or biased results. When such conditions are not met, alternative methods must be considered.

As a result, alternative methods, often referred to as non-parametric techniques, must be considered. 
These methods rely on the rank $R_t$ of the observations $x_t$ rather than their actual values, making them robust and applicable to a wide range of data sets. Since non-parametric tests do not assume a normal distribution, they are highly reliable. For example, the Kruskal-Wallis $H$ test and the Wilcoxon test are effective tools for comparing two or more population probability distributions from independent random samples. However, these techniques are not always suitable for time series data, which often require specialized methods tailored to their unique characteristics.

To address these challenges, ordinal pattern methods provide a robust alternative. Instead of analyzing the absolute values of a time series, these methods focus on the order relationships among consecutive data points.

This approach effectively captures the underlying dynamics of complex systems and offers several advantages.

The ordinal pattern-based method has become a widely used tool for characterizing complex time series. Since its introduction nearly twenty-three years ago by Bandt and Pompe in their foundational paper \cite{PhysRevLett.88.174102}, it has been successfully applied across various scientific fields, including biomedical signal processing, optical chaos, hydrology, geophysics, econophysics, engineering, and biometrics. It has also been used in the characterization of pseudo-random number generators.
   
The Bandt and Pompe method successfully analyzes time series by transforming them into ordinal patterns, constructing a histogram, and computing Shannon entropy, making it robust against outliers and independent of predefined models.

Later, Rosso~\cite{Rosso2007} introduced an additional dimension to this analysis Statistical Complexity derived from the same histogram of causal patterns.

\section*{Introduction to Ordinal Pattern Analysis}

Ordinal patterns are a non-parametric representation of real-valued time series and ordinal patterns are transformations that encode the sorting characteristics of values in $\mathbb{R}^D$ into $D!$ symbols, where $D$ represents for the ''Embedding Dimension'' and usually ranges between three to six. 
One of the possible encoding is the set of indexes that sort the $D$ values in non-decreasing order.
 

To illustrate this idea, let $\bm{x}=\{x_1,x_2, \dots, x_{(n+D-1)}\}$ 
be a real valued time series of length $n+D-1$ without ties. 
As stated by Bandt \& Pompe, if the $\bm{x}$ takes infinitely many values, it is common to replace them with a symbol sequence $\bm{{\pi}}=({\pi}_1, {\pi}_2,\dots, {\pi}_n)$.
consisting of finitely many symbols and then compute the entropy from this sequence. 
The corresponding symbol sequence naturally emerges from the time series without requiring any model assumptions. We compute
$\bm{{\pi}}=({\pi}_1, {\pi}_2,\dots, {\pi}_n)$ symbols from sub-sequences of embedding dimension $D$. 
There are $D!$ possible symbols: $\pi_j \in \bm{{\pi}}=({\pi}^1, {\pi}^2,\dots, {\pi}^{D!})$. 
The histogram of proportions $h=(h_1,h_2,\dots, h_{D!})$ in which the bin $h_\ell$ 
%%% ACF Whenever possible, use \ell instead of the letter ell in mathematical mode
is the proportion of symbols of type $\pi^\ell$ of the total number of symbols. 
For convenience, we will model those symbols as a $k$ dimensional random vector where $k=D!$.

Consider a series of $n$ independent trials in which only one of $k$ mutually exclusive events ${\pi}^1, {\pi}^2,\dots, {\pi}^k$ is observed with probability $p_1, p_2, \dots, p_k,$ respectively, where $p_\ell \geq 0$ and $\sum_{\ell=1}^{k} p_\ell=1$. 
Suppose that $N=(N_1, N_2, \dots, N_k)$ is the vector of random variables that, with $\sum_{\ell=1}^{k} N_\ell=n$, counts how many times the events ${\pi}^1, {\pi}^2,\dots, {\pi}^k$ occur in the $n$ trials. Then, the joint probability distribution of $N$ is
%%% ACF NEVER leave a blank line between the paragraph and the equation that follows
\begin{equation}
	\Pr\big(N=(n_1,n_2,\dots, n_k)\big) = n! \prod_{\ell=1}^{k} \frac{p_\ell^{n_\ell}}{n_\ell !}, %%% ACF Propagate the changes
\end{equation}    
where $n_\ell \geq 0$ and $\sum_{\ell=1}^{k} n_\ell=n$ .


\section*{Problem Statement}
%To illustrate this concept, we consider food preferences among individuals as an example of time series data. Meal choices vary widely among people, reflecting their unique tastes and priorities. For instance, when selecting main meals such as beef, pork, chicken, fish, and vegetables, each person has distinct preferences. By analyzing these choices, we can uncover intriguing patterns that highlight the diversity in individual selections.


To illustrate this concept, imagine tracking the mean monthly humidity in Wellington. You want to analyze how humidity changes throughout the year. By examining this data, you can uncover interesting patterns that highlight the variations in humidity across different months. 

%Imagine two individuals with different meal preferences. Person 1 enjoys pork the most, followed by beef, fish, chicken, and vegetables. In contrast, Person 2 prefers chicken first, followed by fish, vegetables, beef, and pork. A simple graph illustrating these preferences would show that their choices do not overlap significantly, emphasizing their unique tastes.

%Expanding this analysis to a larger group allows us to observe even more diverse and complex patterns. Each individual's preference forms a unique data point, and collectively, they create a rich tapestry of variation. This exploration provides valuable insights into how food choices differ across individuals and groups, making the study both meaningful and captivating.

\begin{table}[hbt]
	\centering
	\begin{tabular}{lc}
		\toprule
		 Month & Mean of relative humidity \\
		\midrule
		January & 77.3 \\ 
		February & 81 \\
		March & 82.4 \\
		April & 81.7 \\
		May & 83.6 \\ 
		June & 85.6 \\
		July & 84.4 \\
		August & 83.1 \\ 
		September & 78.8 \\
		October & 79.6 \\
		November & 78.2 \\
		December & 78.8 \\
		\bottomrule
	\end{tabular}
	\caption{Mean monthly humidity variations in Wellington throughout the year}
	\label{tab:humidity}
\end{table}

Mean monthly humidity in Wellington is shown in Figure~\ref{fig:humidity}  
\begin{figure}[hbt]
	\centering
	\includegraphics[width=0.6\textwidth]{humidity graph}
	\caption{Mean Monthly humidity in Wellington}
	\label{fig:humidity}
\end{figure}

We can convert this actual data into ordinal patterns. 
To do this, for each month, we determine the order of the humidity values rather than their actual magnitudes. 
Each three-time-point sequence (which can be adjusted based on preference) is converted into an ordinal pattern.
This ``embedding dimension'' usually varies between $3$ and $6$, but any dimension is possible.
The conversion can be made in any way that uniquely maps the sorting properties of the subsequence into a symbol.
%%% ACF Summarise and present in your own words from Section 3
%@Article\{AsymptoticDistributionofEntropiesandFisherInformationMeasureofOrdinalPatternswithApplicationsa,
%author    = \{Rey, A. A. and Frery, A. C. and Gambini, J. and Lucini, M.\},
%journal   = \{Chaos, Solitons \textbackslash{}\& Fractals\},
%title     = \{Asymptotic distribution of entropies and \{F\}isher information measure of ordinal patterns with applications\},
%year      = \{2024\},
%issn      = \{0960-0779\},
%month     = nov,
%pages     = \{115481\},
%volume    = \{188\},
%doi       = \{10.1016/j.chaos.2024.115481\},
%groups    = \{Journal Article\},
%publisher = \{Elsevier BV\},
%\}


%%% ACF Use only the lines provided by the booktabs package
%%% ACF The table is confusing: "# of patterns" should be just $t$, Ordinal Patters should be Ordinal pattern, and you should assign a type to each, i.e., \pi^1, \pi^2 etc

\begin{table}[H]
	\centering
	\begin{tabular}{lcr}
		\toprule
		 $t$ & Mean Humidity sequence & Ordinal Pattern \\
		\midrule
	1 & (77.3,81,82.4) & (123) $=\pi^1$\\ 
	2 & (81,82.4,81.7) & (132) $=\pi^2$\\
	3 & (82.4,81.7,83.6) & (213) $=\pi^3$\\
	4 & (81.7,83.6,85.6) & (123) $=\pi^1$ \\
	5 & (83.6,85.6,84.4) & (132) $=\pi^2$\\ 
	6 & (85.6,84.4,83.1) & (321) $=\pi^6$\\
	7 & (84.4,83.1,78.8) & (321) $=\pi^6$\\
	8 & (83.1,78.8,79.6) & (312) $=\pi^5$\\ 
	9 & (78.8,79.6,78.2) & (231) $=\pi^4$\\
	10 & (79.6,78.2,78.8) & (312) $=\pi^5$\\
		\bottomrule
	\end{tabular}
	\caption{Ordinal Patterns}
	\label{tab:op}
\end{table}

As shown in Table~\ref{tab:op}, we have six mutually exclusive events,   
$({\pi}^1, {\pi}^2,\dots, {\pi}^{6})= {(123),(132),(213),(231),(312),(321)}$ respectively. 
The probability distribution of the mean humidity is calculated based on ordinal patterns as given below.
\begin{equation}
	\hat{p_i}=\frac{\#\{\pi_j \in \bm{\pi}: \pi_j=\pi^{i}\}}{n}; 1\le i \le 6
\end{equation}
where $\hat{\bm{p}}=(\hat{p_1}, \dots ,\hat{p_6})$

\begin{table}[hbt]
	\centering
	\begin{tabular}{lc}
		\toprule
		Notation & Probability \\ 
		\midrule
		$p(\pi^1)$ & $\frac{2}{10}$ \\ 
		$p(\pi^2)$ & $\frac{2}{10}$ \\
		$p(\pi^3)$ & $\frac{1}{10}$ \\
		$p(\pi^4)$ & $\frac{1}{10}$ \\
		$p(\pi^5)$ & $\frac{2}{10}$ \\ 
		$p(\pi^6)$ & $\frac{2}{10}$ \\
		\bottomrule
	\end{tabular}
	\caption{Probability function}
	\label{tab:probability_function}
\end{table}

We construct the histogram of proportions $h=(h_1,h_2,h_3,h_4,h_5,h_6)$, where each bin $h_\ell$ represents the proportion of symbols of type $\pi^\ell$ out of the total six symbols. The histogram graph is shown Figure~\ref{fig:histogram}.

\begin{figure}[hbt]
	\centering
	\includegraphics[width=0.6\textwidth]{frequency histogram}
	\caption{Histogram of proportions of the observed patterns according to Table~\ref{tab:probability_function}.}
	\label{fig:histogram}
\end{figure}
%%% ACF Always add information as in the following line
%%% The above plot was produced by lines 41-57 of file 'Code/R/humidity_example.R'

This example explains how time series data can be converted into ordinal patterns and how the probability distribution function can be calculated from these patterns. Chapter Two will review the literature on ordinal pattern analysis, while Chapter Three will expand on this concept by exploring the characterization of time series. It will also cover the computation of two key descriptors entropy and complexity from the resulting histograms. Additionally, Chapter Three will outline the main ideas and objectives of this research project.      







\chapter{Literature Review}\label{C:lit}

This chapter mainly focus to discuss about the literature review related to this research work. We are analyzing Bandt and Pompe based research work and we found 3

\section{Preliminaries}
sssss


\chapter{Literature Review}\label{C:lit}

The analysis of complex time series has long relied on both time-domain and frequency-domain techniques. 
However, traditional methods often fall short in capturing nonlinear dynamics or are limited by strict assumptions such as stationarity and Gaussianity. 

The ordinal symbolic approach introduced by Bandt and Pompe in 2002 marked a significant theoretical advance by enabling robust, model-free characterization of time series. 
Their approach, rooted in information theory, involves converting segments of time series data into symbols based on the ordinal (rank) relationships among the data points.
These symbols are called "ordinal patterns." After computing all the symbols, their relative frequencies are used to estimate the probability distribution of ordinal patterns.

From this distribution estimate, two key descriptors entropy and complexity are calculated to characterize the time series: the scaled Shannon entropy, now widely known as permutation entropy, and the statistical complexity.

This chapter is divided into three main sections. Section~\ref{Sec:Onset} presents a brief overview of the area, focusing on what we consider the four seminal papers. Section~\ref{Sec:ResearchQuestion} discusses the research question and the motivation for conducting the bibliometric analysis. The final section, Section~\ref{Sec:BiblioIntro}, highlights the importance of bibliometric analysis and presents the results obtained from references that cite the Bandt and Pompe methodology and other related topics based on our research focus.  
%\textcolor{red}{COMPLETE WITH ONE LINE}


\section{The Onset of the Entropy-Complexity Plane}\label{Sec:Onset}

This section discusses the emergence of the entropy-complexity plane. This topic is presented as a central theme because it reflects the foundation of our main research focus and illustrates how it has evolved over time into the current approach to time series analysis based on the concept of ordinal patterns. 

López-Ruiz et al.~\cite{lopez1995statistical} to capture the structure of a system:
the product between the entropy and a distance between the estimated model and a non-informative model is an interesting way of measuring complexity.
Lamberti et al.~\cite{lamberti2004intensive}, using that idea, proposed using the Euclidean distance between the measured probability function and the uniform distribution.
Rosso et al.~\cite{EEGAnalysisUsingWaveletBasedInformationTools} discussed using other distances, proposed the Jensen-Shannon distance, and used it jointly with the scaled Shannon entropy to form a bivariate feature.
They mapped this feature into the so-called ``Entropy-Complexity Plane,'' devising  a powerful diagnostic tool to distinguish between different dynamical regimes, such as chaos, noise, and periodicity.
Further, Martin et.al.~\cite{Martin2006} discussed the boundaries of this generalized statistical complexity measure. 

In the following, we present the research question and the motivation for continuing this research. 

\section{Research Question and Motivation}\label{Sec:ResearchQuestion}

The primary research question guiding this study is:
\begin{quote}
	\textit{How can confidence intervals for generalized entropy measures (Shannon, Tsallis, Rényi, Fisher information measure) and their associated complexity metrics be used to improve the robustness and discriminative power of time series clustering techniques?}
\end{quote}

We are motivated to conduct a literature review to confirm the relevance of our research areas in relation to the research question. Our aim is to determine whether other researchers are engaging with similar types of questions. Additionally, we seek to verify whether there is a strong focus on practical applications within this topic. 


\section{The Bibliometric Analysis: data collection, tools and background}\label{Sec:BiblioIntro}

Bibliometric analyses provide a quantitative approach to reviewing and mapping the intellectual structure of a research field. By systematically analyzing citation patterns, author collaborations, and keyword co-occurrences, they help identify key themes, research trends, and emerging topics. This method ensures objectivity, reveals key contributions, and offers a structured overview of intellectual development, making it a valuable tool for systematic literature reviews.

In this study, we conducted a bibliometric analysis using the Bibliometrix package in R and its user-friendly web interface Biblioshiny~\cite{Aria2017} focusing on literature related to ordinal patterns, permutation entropy, and complexity measures in time series analysis. 

Section~\ref{Subsec:Dataextraction} discusses the data extraction process using the Bibliometrix package in R.
%%% ACF Cite it

\subsection{Conceptual Structure: Data extraction and Summary Statistics}\label{Subsec:Dataextraction}

%%% ACF Make it clear that this is the data extraction phase
Scopus-indexed references that cited the seminal work by Bandt and Pompe, along with other references relevant to our research topic, were collected on June 9, 2025.
Based on these reference files, we analyzed a dataset consisting of 4125 reference files spanning the years 1993 to 2025. %%% ACF If you used papers that cited B&P, how can there be papers from 1993?
%%% ACF REVISE REMOVE REDUNDANT PARTS OR MOVE THEM TO WHERE THEY BELONG
%%% ACF USE \siunitx
The descriptive analysis of the dataset revealed a total of 4063 usable documents (out of 4125; the others had missing data and were removed), sourced from 1317 
%%% ACF What are they? publication sources refer to the outlets where research outputs (like journal articles, conference papers, books, etc.) are published. 
publication sources. 
The dataset shows an annual growth rate of \SI{18.15}{\percent}, %%% ACF??? indicates the percentage increase or decrease in the number of items within that dataset over a specific year. This metric helps researchers understand the trend of scholarly activity in a particular field, showing whether research output is expanding, contracting, or remaining stable. 
involving 7254 authors, 
123 single-authored documents, 
\SI{27.32}{\percent} international co-authorship, 
with an average of 4.28 co-authors per document. 
The author keywords totaled 7667, 
with an average document age of 5.7 years, %%% ACF???the mean age of all documents (e.g., articles, books) included in a study, typically measured in years from their publication date. It provides insights into the lifespan of research within a specific field or domain and can indicate how quickly research findings are disseminated and cited. A lower average age suggests a more rapidly evolving field, while a higher average age might indicate more established or mature research areas. 
and 22.6 citations per document. 

\subsection{Thematic Map Analysis}

Thematic map helps to understand the research direction and the relevant topics for the future studies. Therefore, we motivated to analyse it. The axes of the thematic map depicts the strength of their internal (density), which reflects inter-cluster growth, and external (connectivity) relevance or significance of the study in a particular area (centrality). 

Figure~\ref{fig:ThematicMap} illustrates the thematic map derived from author keywords.

\begin{figure}[H]
	\centering
	\includegraphics[width=\textwidth]{ThematicMap-2025-06-20}
	\caption{The Thematic Map generated by Bibliometrix.}
	\label{fig:ThematicMap}
\end{figure}

The map is divided into four quadrants based on two dimensions: centrality (relevance) and density (development). 
A right upper quadrant representing motor themes that are both well developed and highly interconnected areas of research, such as EEG, machine learning, permutation entropy, complexity, and other types of entropy. This theme is also at the core of current research, particularly in areas such as biomedical signal processing and nonlinear time series analysis. The consistent presence of entropy-based measures and machine learning highlights the interface between theory and practice.
%%% ACF What are Motor Themes? 

Emerging or declining themes like image encryption reflect peripheral or potentially declining research interests, while chaos and semiconductor lasers hold theoretical interest, but its practical integration appears limited.  

A right down quadrant depicts basic themes indicates that opportunities for further theoretical and methodological advancement. 
Research fields such as fault diagnosis, feature extraction and rolling bearings are at the center of the map. Its moderate centrality and density mean that it is still active and is subject to evolving research fields. These topics are closely linked to engineering and diagnostic applications. These are important and growing areas which require further methodological refinement and integration.
%%% ACF What are Basic Themes... and so on

The size of each circle further represents the frequency of the topic based on keyword occurrences associated with the publications. The clustering structure shows that entropy-related measures (such as the entropy of a permutation and the complexity of a system) are gaining ground not only in theory but also in practical applications such as EEG and machine learning.
%%% ACF This is NOT a thematic analysis. This is the output of bibliometrix. The thematic analysis starts from this evidence

From these results, we conclude that our research focus on entropy and complexity of permutation is relevant for many studies. These are basic concepts which are widely used in the literature, but which offer considerable potential for further development. Moreover, this thematic map shows that research is strongly focused on entropy-based methods, machine learning and biomedical applications, with fault diagnosis and feature extraction emerging as promising intermediate topics for further discuss.

%\textcolor{red}{From this results, we conclude that\dots}

\subsection{Factorial Analysis}
To identify the broad overview of the main research topics, we analyze the conceptual structure map. In this case we considered all keywords which are automatically generated by indexing databases. The shaded polygon in Figure~\ref{fig:factorialMap}  outlines the conceptual space defined by the most distinctive keywords. The X-axis (Dim 1) and the Y-axis (Dim 2) are the first two dimensions of the factorial space, and explain the largest differences in the co-occurrence of the keywords.

\begin{figure}[H]
	\centering
	\includegraphics[width=0.9\textwidth]{FactorialMap}
	\caption{Conceptual Structure map generated by Bibliometrix.}
	\label{fig:factorialMap}
\end{figure}
%%% ACF Interpret: what do the axes mean? What does the point size encode?

The conceptual structure map reveals four major clusters:

The first cluster (upper right) includes theoretical concepts such as chaos theory, chaotic systems, Lyapunov methods, ordinal pattern, information theory, complex networks, and permutation entropy, forming the theoretical core of the research area.

The second cluster (lower right) includes practical applications such as feature extraction, empirical mode decomposition, variational mode decomposition, fault diagnosis, fault detection, and failure analysis, emphasizing the applied relevance of complexity-based time series analysis.

The third cluster (center) comprises terms related to biomedical signal processing and machine learning, including biomedical signal processing, EEG, support vector machine, classification, and machine learning, indicating the multidisciplinary applications of entropy measures.

A fourth, smaller cluster (left) includes clinical and physiological study keywords such as clinical article, controlled study, adult, male, female, humans, and physiology.

The conceptual map thus provides further evidence of the diverse applications and theoretical development surrounding ordinal patterns and complexity measures, supporting the originality and relevance of the present research.

%\textcolor{red}{MAKE YOUR CONCLUSIONS HERE. WHICH TOPICS AND WHY?}
\subsection{Conclusion and Justification of Research Focus}

Based on the thematic map and factorial analysis, research topics are categorized into five clusters and we will structure our review of the literature based on this clusters:

\begin{itemize}
	\item Permutation entropy, Complexity and other types of Entropy (section~\ref{Sec:ReviewTopicPE});
	\item EEG and Machine learning approach (section~\ref{Sec:ReviewTopicEEG});
	\item Fault diagnosis and failure analysis (section~\ref{Sec:ReviewTopicFault});
	\item Chaos other research works (section~\ref{Sec:ReviewTopicOthers});
	%\item Image encryption  (section~\ref{Sec:ReviewTopicImage}).
\end{itemize}

\section{Permutation entropy, Complexity and other types of Entropy}\label{Sec:ReviewTopicPE}
This cluster is associated with the keywords time series analysis, statistical complexity, nonlinear dynamics, ordinal patterns, entropy, and information theory. Although ordinal pattern based research have gained wide recognition as powerful tools for nonlinear time series analysis, with applications ranging from biomedical signal processing to cyber-physical systems, their theoretical and practical challenges remain unsolved \cite{Keller2017, Zanin2012}. 

Li et.al.~\cite{Li2015e} asses the complexity of short-term heartbeat interval series by using distribution entropy. They found that sample entropy (SampEn) or fuzzy entropy (FuzzyEn) quantifies essentially the randomness, which may not be uniformly identical to complexity. Zhang et.al.~\cite{Zhang2018d} uses the ordinal pattern technique to analyze dynamic behaviors such as regular, chaotic, or random patterns. The paper highlights the application of ordinal patterns in various fields, including ecology, finance, and physiology, where assessing variability and complexity is essential. It also emphasizes the advantages of using ordinal models to identify structural differences in time series that may not be detectable through traditional statistical methods.

Several studies have formalized the statistical behaviour of entropy and complexity measures derived from ordinal patterns, and have provided insights into their asymptotic distribution~\cite{Rey2024, Rey2023a} and sensitivity under the Multinomial law~\cite{Rey2023}. Despite these advances, the reliability of these measures in finite sampling conditions, in particular in non-stationary and noisy environments, remains limited~\cite{Borges2023}. 


Applications such as  Internet of Things(IoT) botnet detection~\cite{Borges2023} and  synthetic aperture radar(SAR) structure classification~\cite{Chagas2021a} have demonstrated the discriminability of ordinal-based features, particularly when combined with multiscale analysis; however, these methods often depend on the optimisation of parameters (dimension, delay) and may lack generalizability to a wide range of data. Similarly, the integration of entropy-complexity representations of class separation in time series dynamics~\cite{Borges2022}, although effective in a structured environment, poses problems when extended to real-time or data-scarce scenarios. 
%Data scarcity refers to the situation where there is insufficient data to meet the requirements of a system, particularly in enhancing the accuracy of predictive analytics.

The conceptual works linking ordinal complexity to broader ideas, such as the technological singularity~\cite{Modis2022} and the development of artistic expression ~\cite{Sigaki2018} illustrate the richness of the framework, but these studies are often qualitative and lack empirical rigour. 

Entropy-based clustering techniques have revealed evolving efficiency patterns in cryptocurrency markets~\cite{Sigaki2019}. The reliance on sensitive parameters and lack of standardized benchmarks highlight the need for more robust and interpretable methods to track market maturation reliably.

Moreover, white noise testing using entropy-complexity plane~\cite{Chagas2022a} and the use of ordinal properties in compressor signal diagnostics~\cite{Barbosa2024} demonstrate the methodological versatility of ordinal approaches. , the lack of uniform and reliable reference points prevents cross-domain comparison. Therefore, while ordinal methods provide a mathematically elegant and computationally efficient basis for obtaining information from complex signals, further research is needed to improve their statistical robustness, interpretability and adaptability to the challenges of the real world.
%%% ACF Do you need to describe again what the method is?
%Zanin et.al.~\cite{Zanin2012}  discuss the development and applications of permutation entropy, a method that captures the temporal structure and complexity of time series by analyzing the order relations among data points. 
%%% ACF Who is "He"?
%He focuses solely on probability distributions, permutation entropy incorporates temporal dynamics, making it particularly effective for studying chaotic and complex systems. 

\section{EEG and Machine learning approach}\label{Sec:ReviewTopicEEG}

%%% ACF Are EEG, epilepsy etc. "applications of entropy measures"?
This cluster is based on multidisciplinary studies involving entropy measures applied to areas such as EEG, epilepsy, classification, heart rate variability, deep learning methods, and nonlinear analysis. An analysis of the author keywords indicates that many of these studies are centered on applications in biomedical signal processing. 
%%% ACF Make the citation
Acharya et al. ~\cite{Acharya2015a, Acharya2015, Acharya2017, Acharya2016, Acharya2019,  Acharya2017a, Acharya2018} have made significant contributions to biomedical signal processing by developing and evaluating advanced automated diagnostic tools across a wide range of clinical applications. Their work includes the use of entropy measures to detect epilepsy and heart disease from EEG and ECG signals, as well as the use of nonlinear dynamics to enhance the detection of sleep stages and the characterisation of focal EEG signals. In the cardiovascular field, they have performed comparative studies on the localization of myocardial infarction using various ECG leads and have developed empirical decomposition methods to identify congestive heart failure from cardiac
%%% ACF Sort them so they appear in a sequence
signals. Another study by Lajnef et.al.~\cite{Lajnef2015} revealed that an automated approach to the classification of sleep stages using a multi-class support vector machine (SVM) based decision tree approach. The proposed method uses physiological signals (such as EEG, EOG, and EMG) to effectively classify the different stages of sleep.

\section{Fault diagnosis and failure analysis}\label{Sec:ReviewTopicFault}
This section primarily relates to the engineering applications of complexity-based time series analysis. The author keywords most commonly used to categorize this cluster include fault diagnosis, feature extraction, rolling bearing, support vector machine, variational mode decomposition, multiscale permutation entropy, dispersion entropy, rotating machinery, and empirical mode decomposition. 

Recent studies in bearing fault diagnosis often use entropy-based methods. Multiscale permutation entropy (MPE) and dispersion entropy are two popular techniques. These methods help detect complex changes in signals under different working conditions. For example, using variational mode decomposition with weighted entropy features helps extract useful information from non-stationary vibration signals. This improves how well faults can be classified~\cite{Lei2024a}. The weighted multiscale entropy method also works well by focusing on important frequency parts of the signal~\cite{Minhas2021a}. Self-adaptive hierarchical multiscale fuzzy entropy is also applied in bearing fault diagnosis. It makes fault detection easier without needing many manual settings~\cite{Yan2021}.  Composite multiscale fluctuation dispersion entropy can detect small fault signs even in noisy signals~\cite{Gan2019}. Some methods combine data decomposition with multiscale permutation entropy to better handle complex, changing systems~\cite{Yasir2018}. These improved entropy methods are also used in medical signal analysis, such as ECG or EEG, showing they work in other areas~\cite{Azami2017, HumeauHeurtier2015}. Dispersion entropy is known for being fast and good at finding small signal changes~\cite{Rostaghi2019,Rostaghi2016}, whereas multiscale permutation entropy still has issues. It can be affected by the length of the signal, noise, and it can be slow to compute~\cite{HumeauHeurtier2015, Zheng2014c}. Multiscale Permutation Entropy (MPE) with the Natural Visibility Graph (NVG) to enhance the fault diagnosis of rolling bearings by capturing both the dynamic complexity and structural features of time series method is proposed by Ma et.al.~\cite{Ma2025}. However, the method may still face limitations related to computational cost, parameter sensitivity, and the requirement for relatively long and noise-free signals to ensure reliable multiscale analysis.Therefore, more research is needed to make these methods faster, better with noise, and easier to use in different fault diagnosis tasks.

\section{Chaos and Other research works}\label{Sec:ReviewTopicOthers}
Research works related to chaos, semiconductor lasers, and image encryption are discussed in this category. 
A self-synchronous chaotic stream cipher, designed to resist active attacks and limit error propagation during image transmission, is a novel technique for image encryption.~\cite{Fan2018}. The 2D discrete wavelet transform, Arnold mapping, and a four-dimensional hyper-chaotic system with positive Lyapunov exponents are used to enhance the security and complexity of the encryption method. The advancement of chaos-based encryption and intelligent video security techniques in modern information systems has been demonstrated through a successful hardware implementation that transforms non-chaotic systems into chaotic ones, significantly enhancing unpredictability for secure communication. In addition, temporal action segmentation for video encryption has been analyzed to optimize computational resources and improve data protection~\cite{Gao2024,Liu2024k}
%\section{Image Encryption}\label{Sec:ReviewTopicImage}
\chapter{Future Works}\label{C:futw}

As a short summary, we have completed the necessary preliminaries studies on various topics such as:
\begin{itemize}
    \item Entropy
    \item Complexity
    \item Entropy Complexity Plane
    \item Confidence interval
    %\item Multinomial modal.
\end{itemize}

We have also examined key research articles by Bandt and Pompe \cite{PhysRevLett.88.174102}, along with an overview of the area focusing on four seminal works. These include:
\begin{itemize}
	\item López-Ruiz et al. \cite{lopez1995statistical}, who introduced the concept of statistical complexity;
	\item Lamberti et al. \cite{lamberti2004intensive}, who applied López-Ruiz's idea using the Euclidean distance;
	\item Rosso et al. \cite{EEGAnalysisUsingWaveletBasedInformationTools}, who proposed the entropy-complexity plane as a diagnostic tool; and
	\item Martin et al. \cite{Martin2006}, who defined the theoretical boundaries of this generalized statistical complexity measure.
\end{itemize}

In addition, we reviewed recent work by Rey et al. \cite{Rey2025,Rey2023a,Rey2023}, which investigates the statistical properties of entropy derived from ordinal patterns, including the asymptotic distribution under the Multinomial law and the behavior of permutation entropy. 

As a case study, we computed the Shannon entropy, statistical complexity, and their associated asymptotic variances based on the probability distribution of ordinal patterns. Using these results, we derived confidence intervals for both entropy and complexity. The analysis was further visualized using the entropy–complexity plane, offering insights into the underlying system dynamics. All computations were performed using two large-sample datasets under the asymptotic distribution, as detailed in Chapter~\ref{C:aim}, Section~\ref{Sec:CaseStudy}. 


The formulas and procedures used to analyze the case study are summarized as follows:
\begin{itemize}
	\item Calculate the Shannon entropy of the time series.
	\item Calculate the statistical complexity.
	\item Estimate the asymptotic variance for Shannon entropy.
	\item Construct confidence intervals for entropy.
	\item Plot the results in the entropy–complexity plane.
	\item Divide the data into batches (batch size = 10,000).
	\item Repeat the above calculations for each batch.
	\item Graphically represent the results of the two time series across batches in the entropy–complexity plane.
	\item Finally, the results are analyzed for time series clustering, as shown in the final output in Figure~\ref{fig:EntopyComplexity Plane}
\end{itemize}

%%% ACF Do not repeat what has already been defined.
Asymptotic distribution of normalized Shannon entropy $H(\mathbf{p})$ was derived under the assumption of independent ordinal patterns, following the Multinomial law. As a foundational step, we use the normalized Shannon entropy formula: 
\begin{equation}
	H(\mathbf{p})=-\dfrac{1}{\log k}\sum^{k}_{\ell=1}p_{\ell} \ln{p_{\ell}}.
\end{equation}
Where, $k=D!$ is the number of possible ordinal patterns. To evaluate statistical complexity, we compute the Jensen–Shannon divergence between the histogram of proportion $\mathbf{p}$ and the uniform probability function $\mathbf{u}=(1/k, 1/k, \dots, 1/k)$, defined by:  
\begin{equation}
	Q'(\mathbf{p,u})=\sum^k_{\ell=1} p_\ell\log\dfrac{p_\ell}{u_\ell}+u_\ell\log\dfrac{u_\ell}{p_\ell}.
\end{equation}

This disequilibrium measure is normalized using:
\begin{equation}
	Q=\dfrac{Q'}{\max{(Q')}},
\end{equation}
where $\max(Q')$ is defined as follows
\begin{equation}
	\max(Q')=-2 \left[\dfrac{k+1}{k}\log(k+1)-2\log(2k)+\log k\right].
\end{equation}

The statistical complexity is then calculated as:
\begin{equation}
	C=HQ,
\end{equation}
where both $H$ and $Q$ are normalized quantities, therefore $C$ is also normalized.   

Then the entropy-complexity plane, which is a two-dimensional representation used to graphically represent the results. 

As a key component of our research, we also calculated the asymptotic variance of the Shannon entropy estimator. The estimated normalized entropy based on sample proportions $\widehat{\bm{p}}$ is: 
\begin{equation}
	H_s(\widehat{\bm{p}})=-\dfrac{1}{\log k}\sum_{\ell=1}^{k}\widehat{p_\ell}\log\widehat{p_\ell}.
\end{equation}

The corresponding asymptotic variance under the Multinomial model is given by:
\begin{equation}
	\widehat{\sigma}^2_p=\dfrac{1}{n}\sum_{\ell=1}^{k}p_\ell(1-p_\ell)(\log p_\ell+1)^2-\dfrac{2}{n}\sum_{j=1}^{k-1}\sum_{\ell=j+1}^{k}p_\ell p_j(\log p_\ell+1)(\log p_j+1).
\end{equation}
where $n$ is the sample size. From this variance, we derive confidence intervals for entropy, which are used to assess uncertainty in the entropy-complexity plane. The asymptotic distribution of statistical complexity under the Multinomial law is:
\begin{equation}
		C[\widehat{\bm{p}}]=H[\widehat{\bm{p}}]Q[\widehat{\bm{p}}].
\end{equation}

%As defined by Rey et.al.~\cite{Rey2025} the asymptotic distribution of $C[\widehat{\bm{p}}]$ by a normal law, with mean of Shannon entropy ($\mu_C$) and variance (${\sigma}^2_C$)is given by:
%\begin{equation}
%	\mu_C=\dfrac{\max(Q'){\sigma_Q}{\sigma_p}}{n({\delta_Q}{\delta_H}+\rho)},
%\end{equation} 
%\begin{equation}
%	{\sigma}^2_C=\dfrac{{\max(Q')}^2{\sigma^2_Q}{\sigma^2_p}}{{n^2}({\delta^2_Q}+{\delta^2_H}+2\rho \delta_Q \delta_H+1+{\rho}^2)},
%\end{equation}
%where $\rho$ is the correlation coefficient bet ween normalized Shannon entropy and normalized disequilibrium.
%Further, 

%$\delta_Q=\dfrac{n\times \text{asymptotic mean of the Jensen Shannon divergence}(\mu_Q)}{\text{asymptotic variance of disequilibrium}(\sigma_Q)}$,

%$\delta_H=\dfrac{n\times \text{asymptotic mean of the Shannon Entropy}(\mu_C)}{\text{asymptotic variance of Shannon entropy}(\sigma_p)}$.

%Assuming that $n$ is sufficiently large, $\delta_Q$ and $\delta_H$ tend to infinity, we can calculate asymptotic distribution of the complexity.

This approach will be further analyzed, as described in the following objectives, to evaluate the accuracy of the results.

This proposal has three objectives in order to continue this research work.
\begin{itemize}
	\item Define a data base of time series for clustering, i.e., finding similar time series. 
	\item Extract all the features we know from their Bandt \& Pompe symbolization (Shannon, Tsallis and Renyi entropies, Fisher information measure, complexities, and the available confidence intervals)
	\item Use those features for time series clustering 
\end{itemize} 


%\chapter{Statistical Properties of Features from Ordinal Patterns}\label{C:StatisitcalProperty}

Although ordinal pattern based methods, such as permutation entropy, have been widely used for nonlinear time series analysis, the statistical properties of the features derived from these patterns, such as their distribution, variance, and confidence intervals remain under-explored and require further theoretical and empirical investigation. Therefore, the purpose of this chapter is to investigate the researchers who worked related to ordinal patterns, what kind of statistical properties of features used for their research work.

Wang et al.~\cite{Wang2025} uses the Generalized Gaussian Distribution (GGD) as the statistical distribution for its proposed entropy method. This is explicitly stated in their methodology, and the method transforms raw vibration signals using the GGD's Cumulative Distribution Function (CDF) to map data into a normalized space (0 to 1).
Jieren Xie et.al.~\cite{bibid} introduces a novel approach that replaces traditional probability distributions with evidence theory, specifically using belief functions and mass assignments to quantify uncertainty in time series analysis. Instead of relying on fixed probabilities, this method assigns basic probability masses to subsets of permutation patterns, capturing both uncertainty and ignorance through belief intervals. The belief permutation entropy (BPE) is calculated using Deng entropy, which generalizes Shannon entropy by incorporating the cardinality of subsets, allowing for a more flexible representation of uncertainty. This framework integrates neighborhood relationships among permutation patterns, enhancing robustness to noise and ambiguity, particularly in short or non-stationary time series.


	
	
%%%%%%%%%%%%%%%%%%%%%%%%%%%%%%%%%%%%%%%%%%%%%%%%%%%%%%%
	
% and of course book style knows about backmatter
% \backmatter caused problems with appendices :-(
% and of course report style doesn't
%%%%%%%%%%%%%%%%%%%%%%%%%%%%%%%%%%%%%%%%%%%%%%%%%%%%%%%
	
	
%\bibliographystyle{ieeetr}
\bibliographystyle{acm}
\bibliography{BearingFaultDiagnosis}
	
	
\end{document}
