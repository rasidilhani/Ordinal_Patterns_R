\documentclass[a0,portrait]{a0poster}
\usepackage{multicol} % This is so we can have multiple columns of text side-by-side
\columnsep=100pt % This is the amount of white space between the columns in the poster
\columnseprule=3pt % This is the thickness of the black line between the columns in the poster
\usepackage{fancybox}
\usepackage[svgnames]{xcolor} % Specify colors by their 'svgnames', for a full list of all colors available see here: http://www.latextemplates.com/svgnames-colors

\usepackage{pdfpages}
\usepackage{cite}
\usepackage{color}
\usepackage{colortbl}
\usepackage[cmex10]{amsmath}
\usepackage{amssymb}
\usepackage{soul}
\usepackage[normalem]{ulem}
\usepackage{booktabs}
\usepackage{color}
\usepackage{placeins}
\usepackage{upgreek}
\usepackage[utf8]{inputenc}
\usepackage{bm,bbm}
\usepackage{bbding}

\usepackage[export]{adjustbox}

%\usepackage{times} % Use the times font
%\usepackage{palatino} % Uncomment to use the Palatino font
\usepackage{graphicx} % Required for including images
\graphicspath{{../../../../Logos/}} % Location of the graphics files
\usepackage[font=small,labelfont=bf]{caption} % Required for specifying captions to tables and figures
\usepackage{amsfonts, amsmath, amsthm, amssymb} % For math fonts, symbols and environments
\usepackage{wrapfig} % Allows wrapping text around tables and figures
\usepackage[framemethod=TikZ]{mdframed}
\usepackage{titlesec} % Modify titles
%\usepackage{fontspec}
%\setmainfont{Calibri}
\newcommand{\ds}{\displaystyle}
\newcommand {\bo}[1]{\textbf{#1}}
\newcommand{\pa}[1]{\left({#1}\right)}
\newcommand{\co}[1]{\left[{#1}\right]}
\newcommand{\ch}[1]{\left\{{#1}\right\}}

\titleformat{\section}{\color{white}\normalfont\Large\bfseries}{\color{white}\thesection}{1em}{\colorbox{SteelBlue}}{}

\setlength{\columnseprule}{0pt}
\mdfdefinestyle{MyFrame}{%
	linecolor=SteelBlue,
	outerlinewidth=2pt,
	roundcorner=50pt,
	innertopmargin=\baselineskip,
	innerbottommargin=\baselineskip,
	innerrightmargin=20pt,
	innerleftmargin=20pt,
	backgroundcolor=white!50!white}
%\input{figconfig}
%\titleformat{command}[shape]{format}{label}{sep}{be %fore-code}{after-code}
\begin{document}
\begin{mdframed}[style=MyFrame]
%----------------------------------------------------------------------------------------
%	POSTER HEADER 
%----------------------------------------------------------------------------------------

% The header is divided into two boxes:
% The first is 75% wide and houses the title, subtitle, names, university/organization and contact information
% The second is 25% wide and houses a logo for your university/organization or a photo of you
% The widths of these boxes can be easily edited to accommodate your content as you see fit
\begin{minipage}[b]{0.33\linewidth}
\raggedright
\includegraphics[width=12cm,valign=t]{NZSA-logo-words-e1505785105112-1.png}
\end{minipage}
%
\begin{minipage}[b]{0.33\linewidth}
\centering
\hfill
\end{minipage}
% \documentclass[options]{class}
\begin{minipage}[b]{0.33\linewidth}
\raggedleft
\includegraphics[width=11cm,valign=t]{"Logo Offshore Standard Landscape Reversed RGB.png"}
\end{minipage}\\

\vspace{3cm}
\begin{minipage}[h]{0.98\linewidth}
\centering \huge \color{SteelBlue} \textbf{Ordinal Pattern Analysis for Early Bearing Fault Detection and Classification in Rotating Machinery -- First Results} \color{Black}\\ % Title
\Large \textbf{Nome S. Sobrenome\textsuperscript{1}, Nome S. Sobrenome\textsuperscript{2}, and Nome S. Sobrenome\textsuperscript{3}}\\ % Author(s)
\normalsize \textsuperscript{1,2} Universidade, Local, Brasil\textsuperscript{1,2} and Universidade, Local, Brasil\textsuperscript{3}\\ %[-0.5cm] % University/organization
email@email.com\textsuperscript{1}, email@email.br\textsuperscript{2} and email@email.com\textsuperscript{3}\\
\end{minipage}
\vspace{0.5cm} % A bit of extra whitespace between the header and poster content

%----------------------------------------------------------------------------------------

\begin{multicols}{2} % This is how many columns your poster will be broken into, a portrait poster is generally split into 2 columns


%----------------------------------------------------------------------------------------
%	ABSTRACT
%----------------------------------------------------------------------------------------

\section{Introduction}

Ordinal Patterns are transformations that encode the sorting characteristics of values in $\mathbbm R^D$ into $D!$ symbols.

They were proposed by Bandt \& Pompe in 2002, and have proven their adequacy at extracting valuable information about the system that produces the data.

One of the possible encodings is associated to the set of indexes that sort the $D$ values in non-decreasing order, for example:
$$
(4.2, 5.1, 7.1, 3.9, 8.6) \longmapsto (2, 3, 4, 1, 5) \longmapsto \otimes \text{ (one of }5!=120 \text{ symbols)}.
$$

A time series $\bm x = (x_1,x_2,\dots,x_{D+n-1})$ can be transformed into a sequence of symbols $\bm \pi=(\pi_1,\pi_2,\dots,\pi_n)$.
Then, we compute $\bm h =(h_1,h_2,\dots,h_{D!})$ the histogram of $\bm \pi$, and from it two descriptors:

\section{Data}\label{section:1}
The data were obtained from the Bearing Data Center and the seeded fault test data at the Case Western Reserve University, School of Engineering. 

The dataset includes ball bearing test data for normal bearings, as well as single-point defects on the drive end and fan end. Data were collected 

at rates of 12,000 and 48,000 data points per second for the drive-end bearing tests, and at 12,000 data points per second for the fanend

bearing tests. Each file includes motor rotational speed, drive-end vibration data, and fan-end vibration data. The variable names in each file

indicate the following:

	\item DE - drive end accelerometer data
	\item FE - fan end accelerometer data
	\item BA - base accelerometer data
	\item time - time series data
	\item RPM - rpm during testing

For ease of use, the data were categorized as Normal Baseline Data, 12k Drive End Bearing Fault Data, 48k Drive End Bearing Fault Data, and Fan

End Bearing Fault Data. The normal baseline data include four motor load levels: 0, 1, 2, and 3, with approximate motor speeds provided in RPM
 
(1797, 1772, 1750, and 1730). The 12k Drive End, 48k Drive End, and 12k Fan End bearing data follow the same motor load levels and speeds. This

research aims to identify faulty machines using the separate time series data.

\section{Methodology}\label{section2}

This study explores the use of ordinal pattern analysis for the early detection and classification of bearing faults, focusing on common fault 

types such as ball, outer race, and inner race defects. Using a publicly available datasets from the Case Western Reserve University Bearing Data 

Center, we demonstrate the effectiveness of ordinal patterns in distinguishing between healthy and faulty bearing conditions.

This research aims to identify faulty machines. Each data file consists of two time series, which we examine using ordinal patterns. We introduce 

distance as a measure of similarity between segments based on their ordinal structure. With specific embedding dimensions, this metric can be used 

to distinguish faulty machines. We apply permutation entropy for rolling bearing fault diagnosis and show that embedding dimensions from 3 to 6 

effectively separate faulty machines. Some machines exhibit white noise, which lies near the lower and upper boundaries. The complexity plane is

used to analyze the results, confirming that dimension 5 yields the best outcomes.

Examples of time series data are shown as follows.



\section{Results}\label{section4}


\section{Conclusions and Future Work}

\section*{Acknowledgements}
%
Este trabalho foi parcialmente suportado por XXXXXX, com recursos  YYYYY.
%----------------------------------------------------------------------------------------
%	REFERENCES
%----------------------------------------------------------------------------------------

%\bibliographystyle{IEEEtran}
%\bibliography{IEEEabrv,references}

%---------------------------------------------------
-------------------------------------
\end{multicols}
\begin{center}
\color{SteelBlue}{New Zealand Statistical Association 2024 Conference}
\end{center}
\end{mdframed}
\end{document}