\documentclass[11pt,a4paper]{article}

\usepackage[margin=1in]{geometry} % Adjust margins
\usepackage{authblk}
\usepackage{hyperref}
\usepackage{setspace}

\setstretch{1.05} % Slightly tighter line spacing
\parskip=4pt      % Compact space between paragraphs
\renewcommand\Authfont{\normalsize}
\renewcommand\Affilfont{\itshape\small}

\title{\vspace{-1cm}Entropy–Complexity Based Clustering of ARMA Simulated Time Series Models\vspace{-0.3cm}}

\author[1]{M.\ H.\ M.\ R.\ S.\ Dilhani}
\author[1]{Alejandro C.\ Frery\thanks{Contact: \href{mailto:alejandro.frery@vuw.ac.nz}{alejandro.frery@vuw.ac.nz}}}
\author[2]{A.\ A.\ Rey}

\affil[1]{School of Mathematics and Statistics, Victoria University of Wellington, New Zealand}
\affil[2]{Laboratorio de Investigación y Desarrollo Experimental en Computación (LIDEC),\\
	Instituto de Tecnología e Ingeniería, Universidad Nacional de Hurlingham (UNAHUR), Argentina}

\date{}

\begin{document}
	\maketitle
	\vspace{-0.5cm}
	
	\begin{abstract}
		
				
		The results highlight the strength of permutation entropy and complexity as feature-based descriptors of time series behavior. This simulation framework validates the reliability of ordinal pattern analysis under diverse structural conditions and underscores its applicability to unsupervised clustering tasks in various domains such as finance, engineering, and bioinformatics.
		
		Ordinal pattern analysis has emerged as a promising alternative to analysing time series data, providing a robust and computationally efficient approach. In this research, ordinal pattern analysis is used to analyse simulated time series under Autoregressive (AR), Moving Average (MA), and Autoregressive Moving Average (ARMA) models. It involves converting time series data into a sequence of symbols that represent ordering relationships among data points within specific time windows. Then, the Shannon entropy and the Statistical Complexity are computed from the histogram of symbols. The method detects small differences between models and sample sizes.
		
		This study investigates a simulation-based experiment into time series clustering using ordinal pattern analysis. The primary research focus was on identifying the distinguishing characteristics of AR, MA, and ARMA models through ordinal patterns features. Simulated time series were generated with two series lengths ($n = 500$ and $n = 1000$), using parameter values chosen to satisfy the stationarity and invertibility conditions of the models.
		
		The experimental design included several variants for each AR, MA, and ARMA model type, with each model representing a distinct parameter setting (see Annex 1). To ensure model accuracy, 100 independent replications were performed for each model configuration.
		
		Ordinal patterns were extracted to compute entropy and complexity, and features were analyzed in the entropy–complexity plane for model discrimination. ARMA(1,1) shows two distinct clusters. AR coefficients $-0.8$ and $0.8$ produce lower entropy and higher complexity, while coefficients $0.1$ and $-0.1$ yield higher entropy and lower complexity. ARMA(2,2) shows overlapping groups across all four models. Clear separation appears by sample size, with larger samples forming more stable clusters (Figures 1 and 2).
		
		The findings highlight the effectiveness of feature-based time series clustering, which uses Shannon entropy and complexity to characterize underling differences among models. The simulation experiment further validates the method’s reliability under varying structural and stochastic conditions. This research establishes a foundation for applying ordinal pattern analysis to unsupervised time series grouping, with potential applications such as finance, engineering, and bioinformatics.
		
	\end{abstract}
	
\end{document}

